%===================================== CHAP 3 =================================

\chapter{Literature Review}
This literature review focuses on academic work that can be adopted to the FTCP. Firstly, Section \ref{Literature_fantasy_real} compares fantasy football to real football. Further, Section \ref{Opt_Models_for_Fantasy_Sports} focuses on existing research concerning optimization models for sport teams, as well as models for Fantasy Sports. Next, in Section \ref{Forecasting_of_future_performance} forecasting of player performance is considered. In addition, this section covers methods for measuring the quality of football teams and home field advantage.

\section{Comparing Fantasy Football to Real Football} \label{Literature_fantasy_real}

A question that arise when discussing the selection of a Fantasy Premier League team is that of whether it's comparable to the selection of a real football team. \cite{Boon} provide an optimization model for selecting the line-up of sports teams, taking account for questions like "How to select an optimal line-up from the set of all candidates in different positions?". Further, \cite{Trninic} discusses the importance of individual player's roles and tasks when selecting the starting line-up, as they focus on finding the position and role most suitable for each player. In addition, they argue that consistency is an important consideration when selecting a line-up. Moreover, \cite{Ozceylan} identifies key performance criteria for each position and suggest a two-phased approach for selecting the players with the greatest contribution to the team. Similarly, \cite{Tavana} propose a two-phase framework for player selection and team formation with regards to determining the collection of individual players that forms an effective team. On the other hand, \cite{Pantuso} propose a stochastic programming approach for composition of football teams. 

\newpar

As for selecting a real football line-up, it is important that the players holds the necessary skills that are compatible to the team's organization of play \citep{Pantuso}. For instance, a quality like leadership is considered important when selecting the captain. In addition, different skills are required for different positions on the field \citep{Boon}. Moreover, it is important that the selected players have a good chemistry and play each other better. Hence, in real football several factors are considered when selecting the team composition. As for the Fantasy Premier League however, it's most important to select players that perform well in terms of Fantasy Premier League points. Thus, the chemistry between the players does not have an impact, as the team mainly consists of players that do not play in the same club. In addition, while leadership is important when selecting the captain of a real football line-up, it is meaningless with regards to FPL line-ups, as the captain is chosen according to the player with the highest expected points. Moreover, player skills like stamina, interceptions and passing have a low impact for the points obtained by a FPL player. For instance, Ngolo Kant\'e, the winner of the PFA Players' Player of the year award in 2017 \citep{Skysports_Kante}, is considered a hard working midfield anchor. During the 2016-2017 season, Kant\'e earned a total of 83 points in FPL, scoring 1 goal, having 1 assist and awarded with 9 yellow cards. Hence, in terms of FPL points this was a poor season for a world class midfielder. However, his contribution was invaluable for Chelsea as they won the Premier League that season. Thus, although Fantasy Premier League is based upon the performances of real football, there are significant differences in terms of selecting the team. While FPL managers should mainly focus on players that score goals, have assists and keep clean sheets, actual football managers have to ensure that the team is composed with the mix of required and compatible skills \citep{Pantuso}.



\section{Optimization Models for Fantasy Sports} \label{Opt_Models_for_Fantasy_Sports}


In recent years, the development of literature in sport analytics has been rapid. Well known journals in statistics, applied mathematics, operations research and economics have published a great number of articles on this field \citep{Coleman}.

\newpar

In terms of operations research, the area of greatest progress is called "Sports Scheduling". This field of research contains problems which involve deciding optimal league season's schedules for different sports. The arena of sports scheduling has existed for over 40 years, but until recently the number of papers has increased significantly \citep{Kendall}. The adaption of operations research can be explained by the growing complexity of playoff structures, league divisions and opposing demands from constituent teams and other stakeholders.  It's use has been investigated in sports including volleyball (\cite{arg_volleyball}), table-tennis (\cite{Mattfeld}; \cite{Knust}), cricket (\cite{Mwright}), basketball (\cite{Wright}; \cite{van_Voorhis}; \cite{Henz}), baseball (\cite{Trick}), Canadian football (\cite{Kostuk}) and softball (\cite{Saur}). In football, sports scheduling is also prominent. \cite{Bartsch} schedules the professional soccer leagues of Austria and Germany. Besides constraints for inner-league requirements and preferences, the European top soccer leagues also has to take into account constraints related to European Cup matches(Champions League, UEFA CUP). They developed models and applied greedy algorithms and branch and bound which yielded reasonable schedules quickly. It resulted in that the schedules were applied by the professional soccer league in both Austria and Germany. Further, \cite{Della} schedules the Italian football league where the schedules also are balanced with respect to additional cable televisions requirements. Further, Chile is perhaps one of the countries where sports scheduling in football has been most present. Since 2005, the Chilean professional soccer has adopted a scheduling system that is based on an integer linear programming model. \cite{Guajardo} generated schedules that gave teams benefits such as higher incomes, lower cost, higher fan attendance and fairer seasons.  Their work was further expanded \citep{Duran}, when they considered challenges for the Chile's Second Division soccer league. Similar scheduling problems are also studied in other countries: Belgium, \cite{Goossens}; Denmark, \cite{Rasmussen}. Finally, \cite{Kendall} addresses the special case of English football fixtures over holiday periods. In England, during Christmas and New Years period up to four matches are played in under ten days. This paper regards the minimization of the travel distances by English football clubs during this period, and generates schedules which have 25 \% lower total travel distances than the fixtures actually used. 



\begin{comment}
1. Bartsch 
2. Della 
3. Chilean 2007 og si at arbeidet ble jobbet videre med i 2012 cite. 
4. Goossens
5. Kendall 
6. Rasmussen 
7. Ribeiro 
\end{comment}
\newpar

In the world of Fantasy Sports the use of operations research is rather limited. Most of the research focuses on NFL Fantasy Draft, which could partly be explained by the higher prizes compared to e.g., fantasy football leagues. Here optimization models and solution methods have been proposed by \cite{Fry}, \cite{Gibson} and \cite{Becker}. Though Fantasy Premier League and NFL Fantasy Draft are both Fantasy Sports, they are essentially different problems. For instance, in NFL Fantasy Draft
only one manager can own a given player - while in FPL this is not limited. Furthermore, for every gameweek each manager is matched head to head where the one with highest points wins - a feature that is not present in FPL. The only similarity that exists is that in each gameweek a fixed number of players in different roles has to be picked, in addition decisions on which players to start and which to not start. This moves into the field of player performance, which will be further discussed in Section \ref{Forecasting_of_future_performance}.


\newpar

In our knowledge, there are no operations research literature which explicitly discusses the FTCP. However, there are articles which could be related. \cite{Mathsports} discuss the common characteristics in Fantasy Sports games, where the common game rules characteristics are the most relevant. From this they present a mixed integer programming model for finding an optimal set of ex-post decisions in Fantasy Sports games under the assumption that all data are known in advance. The model is tested on a fantasy cycling game achieving promising results. Without the gamechips, captain and vice-captain, part of the model can with modification be used for FTCP. However, including the gamechips makes the FTCP more complex and calls for a more sophisticated mathematical model. 

\newpar

\cite{Matthews} are the first to develop a sequential team formation algorithm on Fantasy Premier League. They modelled the Fantasy Team Composition Problem as a belief-sate Markov Decision Problem and solved it using Bayesian Q-learning. Their model produced promising results as it outperformed most human FPL managers; on average it finished around the top 1 \% for 2010/2011 season. They only modelled the wildcard gamechip, with the qualitative assessment of the opportunity to play it in the 8th and 23rd gameweek. Although their method did not specifically use operations research, their results motivate the development of analytic tools for the FTCP. 

\newpar

\cite{Bonomo} presents two optimization models for the Argentinian Fantasy Football. The first model called \textit{a priori} determines lineup and transfers based on prediction of player points, while the second model called \textit{a posteriori} determines the optimal lineup with data known beforehand. The results obtained by the \textit{a priori} model positioned it within top 0.2 \% of all managers. They are the first to develop and solve an optimization model for FTCP, and it encourages the use of operations research on this field. The points in the Argentinian Fantasy Football are to a high degree allocated in the same manner as in the English Fantasy Premier League. They are set by both objective statistics (goals, assists, yellow cards and etc.) and subjective statistics determined by a newspaper. The subjective statistics resemble the Bonus Points System in FPL. Comparing the optimization model to \cite{Mathsports}, many of the constraints can be related, though adjusted for a Fantasy Football. Further, as with the point allocation, the rules in Argentinian Fantasy Football resemble much of the same rules and features of FPL. Nevertheless, there are some major exceptions: 


\begin{itemize}
    \item  Again the gamechips make the FPL unique. As mentioned earlier, these include wildcard, bench boost, free hit and triple captain.  
    \item Over the run of the season, player values can increase or decrease based on how many managers select them. This opens up the possibility of buying players cheap and selling them expensive to buy better performing players. This feature is only present in FPL.
    \item The Argentinian league allows 4 free transfers in each gameweek, while FPL allocate 1 free transfer each gameweek and further the possibility of doing additional transfers which deduct points. By such, a human manager in the Argentinian Fantasy Football has a considerably larger room for error. Each gameweek they can do 3 more free transfers and are to a greater extent guarded from factors such as injury, suspension or players not performing well. A human manager in FPL is more exposed to these factors and faces a choice under greater uncertainty. Firstly, it is likely one has to do more than one transfer if you first decide to do a transfer. For example, due to budget restriction there is a big chance that to afford a more expensive player you have to buy a cheap player in another player position. This requires two transfers, but could also require more. Secondly, it only pays off to take the 4 points hit from an additional transfer if that player in total earn 4 or more points than the player which was transferred out. 
    \item The decisions of which players are selected as captain and vice-captain in each gameweek are only featured in FPL. It calls for human managers to be even better skilled as to pick right players to double their points. 
\end{itemize}

In general, there is a larger number of decisions in FPL and thus increasing the level of complexity and uncertainty. 

\newpar

As can seen from the works above, the amount of academic articles on Fantasy Sports are restricted. As for the use of operations research in Fantasy Sports, the number of articles are even fewer and most of the work is concerned NFL Fantasy Draft. As mentioned earlier, this may be due to the big money prices in this game. No academic works were found on the use of operations research on Fantasy Premier League. This thesis is the first to use operations research in Fantasy Premier League, and is solved using forecasting to predict player performance. 

\begin{comment}
- på samme måte at man har en budget limit finnes det også budget constraints i portfolio. 
- i portfolio kan man investere fraksjoner i assets, mens i ftcp er valgene binary, enten er spilleren med på laget eller ikke. 
- en forskjell er også at i ftcp har man en limit på hvor mange spillere man skal ha totalt, men også en grense på antall spillere i forskjellige roller, mens i portfolje er vanligvis asset classes ikke gitt på forhånd. 
- risk er et viktig og vanlig aspekt i portfølge optimeringen
- skrive om daily fantasy sports artikkelen 
\end{comment}



\begin{comment}

To this date there are no metrics provided on what is the maximum number of points obtainable so far. By solving a model with perfect information, an upper bound of maximum points is obtained and hence useful in evaluating how skillful FPL managers are.

All the post-analysis of FPL to this date are limited to 

Determining the maximum number of points obtainable with perfect information 

To this date there are no 

- igjen er det som skiller vårt problem fra dette at man må ta et valg på kaptein i hver runde og man har muligheten til å bruke gamechips. Også har man med det at spillere kan stige og minke i pris. Samtidig har man i den argentiske ligaen får man kun 4 gratis bytter i hver runde, mens i FPL får man ett gratis bytte i hver runde og muligheten til å gjøre flere points reducing bytter. "Artikkel om varians" presenterer en modell på drafting decisions i NFL og introduserer varians i modellen som et estimat på risk. Man vil til en viss grad minimere eller sete et treshold på risk i forhold til return i en slik oppgave. 

Dette er et aspekt som ikke er blitt vurdert i Argentinsk liga artikkelen, og  


To the best of our knowledge, there has never been developed an optimization model for Fantasy Premier League.

\end{comment}
\begin{comment}


\cite{Fry} define a stochastic dynamic programming model for NFL Fantasy  Draft, focusing on the weekly performance of each player. However, as they state that it is intractable to solve this model by classical solution methods, they suggest an alternative solution approach. By modeling the problem as a dynamic deterministic problem and reducing the sample set of players, they obtain a solution for optimizing a team in the NFL Fantasy Draft. Further, they simplified their model by not considering the actions of the competitors in a NFL Fantasy Draft. A draft Fantasy Sports is quite different from the Fantasy Premier League, as only one manager can own a given player in a draft league. Hence this approach can not be totally applied to FPL. However, as a FPL manager has to make decisions in every gameweek accounting for recent player performance, the dynamic aspect of the model suggested by \cite{Fry} should be considered in this project. 

\newpar

\cite{Gibson} expanded the work of \cite{Fry} by taking account for the competitor's actions. They presented a hybrid metaheuristic for addressing a multi-period stochastic knapsack problem in which the availability of items in future periods is uncertain due to the presence of competitors. This was solved by introducing a beam search algorithm that simulates the actions of the competitors, which are unknown to the decision-makers. As the actions of the competitors does not affect the team selection in Fantasy Premier League, this approach is not considered in this project. 

\newpar

\cite{Matthews} proposed a model for optimization of a Fantasy Premier League team, modelling FPL as a Markov decision problem (MDP). Further, they expanded their model as a belief-state MDP, solving it using Bayesian Q-learning. Their model outperformed multiple FPL managers on predicting the points of a FPL squad over an entire season, as their model finished among the top 20, 000 players for the 2010/2011 season. However, there are lack of information on their approach. The authors were contacted on mail, but without success of gaining more information than provided in the article. Therefore, their model is not considered for this project report. 

\end{comment}

\begin{comment}
There is limited research available on optimization of a Fantasy Premier League team. However, due to the large amount of active NFL Fantasy Sports players, literature is available for both season- and weekly format. \cite{King} suggested a model for predicting points for the quarterbacks in the NFL, using Backward Stepwise regression and Support Vector regression models in his research. Further, he predicted fantasy points by use of Artificial Neural Networks. Similar approaches can be used in order to predict the performance of the Premier League players.
\end{comment}


\section{Forecasting of Future Performance} \label{Forecasting_of_future_performance}
Perhaps the most important skill required in Fantasy Premier League is the ability of predicting future player performances. As stated by \cite{Smith}, fantasy managers have a tendency of ignoring historical statistics when selecting their line-up. As the aim of this thesis is to examine whether an optimization model can outperform human fantasy managers, forecasting of player performance can be of great use. 
\subsection{Individual Player Performance} \label{Forecasting_of_player_performance}
\cite{Yang} predicted the results of the 2016 NBA season by running a least-square linear regression model on players’ individual statistics and the win-ratio of their team, hence predicting the results of the NBA matches based on player performances. As regressors he primarily used well-known statistics such as points scored, assists, rebounds, steals and blocks. These variables were analyzed for the past 20 seasons in order to predict the outcome of the upcoming season. In order to evaluate the impact of the players, he assigned each player with a player efficiency rating, a number representing a player’s effectiveness by a single number. A limitation of Yang’s approach is that the data used mainly focuses on offensive contributions, hence good defending players are not valued properly. Although this paper focuses on predicting a team’s performance and not solely player performance, a similar approach may be used in order to predict the performance of individual Premier League players. 
\newpar
\cite{Hvattum_2015} proposed a top-down rating model for football players, using a regression model to capture how players perform relative to their team mates and to the opposition. They provide a regression based player rating using  a plus-minus player statistics, measuring the number of goals scored minus the number of goals conceded when a player is used by the team. Further, they discount older observation placing greater emphasis on recent performances. As their model did not sufficiently differentiate between players from different divisions or different league systems, their model was improved by \cite{Hvattum_2017}. They extended the model by adding a factor depending on a player's current league and division. 
\newpar
An approach used for rating the players in the Norwegian Tippeligaen was suggested by \citep{vabo}. They applied a binary logistic regression in order to assign values to shots attempted in football, hence calculating the probabilities of a shot resulting in a goal scored. In addition, they developed two Markov game models in order to evaluate all players actions, not only the shots. Their approach is quite advanced and requires heavily datasets delivered by Opta Sports. Although this thesis propose an interesting approach for valuing individual player involvements, the datasets used may be too detailed for forecasting of fantasy points. In addition, the analysis are made post-game, thus there is no forecasting of player performance in this paper. However, the regression variables used and the idea of valuing players using regression are of great interest.
\newpar
As mentioned in \ref{Opt_Models_for_Fantasy_Sports} \cite{Bonomo} propose a mathematical model for an Argentinian fantasy league. In addition to the optimization model, they also provide forecasting of future performance for the individual players. Their approach is quite straightforward and easy to implement. Future points predictions are calculated by taking the average of the past three performances of each particular players. Further, a player's predicted points are weighted by four deciding factors taking account for the upcoming feature:
\begin{itemize}
    \item 1.05 if his team is playing at home or 0.95 if playing away
    \item A value between 0.95 and 1.05, where the value is determined by the position on the league table. The players on the league leading team are weighted by 1.05.
    \item The current performance or situation of the player or the club. This value varies from 0.95 to 1.05.
    \item A starting lineup factor. This factor is set to 1 for players that are expected to start the in the next round and 0 for all other players. 
\end{itemize}
As the problem of modelling the Argentinian Fantasy Premier League resembles the FTCP, this approach can be used for the English Fantasy Premier League as well. Thus, the approach is considered later in this thesis. 

\subsection{Rating Football Teams} \label{Strength_of_football_teams} 
\newpar
A question that arises when one is to predict future performance of individual players is the impact of that particular player's team. It is reasonable to assume that a player will perform better if he plays for a top rated team than he would if he played for one of the weaker teams. Team strengths has been a hot topic in scoreline prediction for several years. Gamblers try to develop scoreline predictions in order to create realistic odds for upcoming matches, hence trying to beat the bookmakers. In addition to development of profitable gambling, an appropriate scoreline prediction could be of great use when one are to predict the future performance of Premier League players. 
\newpar
\cite{Maher} introduced attacking- and defensive strength parameters for each team in order to develop a model calculating the expected goals scored by both a home and away team, where the amount of goals were following a Poisson distribution. This model was further expanded by \cite{Dixon} who were able to calculate probabilities of different scorelines in a given match. Further, they created a strategy for profitable gambling. \cite{Rue} suggested that the attacking- and defensive strengths of a team were time-dependent, and updated the strength estimates using Bayesian methods. In order to do so, they used Markov chain Monte Carlo iterative simulation techniques. 
\newpar
Although the approach suggested by \cite{Maher} has been improved over years, there are other ways of determining the strength of a team. In 1978 the Elo rating was introduced by \cite{Elo}. The rating was initially developed for rating the strength of chess players, but has widely been adopted to different sports, including golf, tennis and football. \cite{Hvattum} used the ELO rating in order to rank teams when predicting match results in football. The ratings appeared to be useful in encoding the information of past results for measuring the strength of a team. However, when used in terms of forecasts it appeared to be considerably less accurate compared to market odds. \cite{vabo} adopted the approach suggested by \cite{Hvattum} for consideration of team strengths as they valued the individual player performance in Norwegian football. In addition, the Elo rating was assessed by \cite{Leitner} along with FIFA/Coca-Cola World Rankings \cite{FIFA} in order to predict the outcome of tournament winners. However, both rating systems were found to be greatly inferior to bookmakers' odds on basis of the outcome of the 2008 Euro Cup results. Hence, it confirms the findings of \cite{Hvattum}. 

\subsection{Home Advantage in Football}\label{HomeFieldAdvantage}
It is no secret that teams have a tendency of performing better when playing at home than when playing away. This topic is of great interest as the effect of home advantage may play an important role when forecasting the performance of a player for an upcoming match. In the following some academic work on home advantage in football is introduced.
\newpar
\cite{Nevill} identifies likely causes of home advantage. They discuss 4 factors thought to be responsible for the home advantage. These factors are categorized under the general headings of crowds, learning, travel and rule factors. When considering the learning factor, they claimed that the players for the home team were used to play at that particular stadium. There was evidence to suggest that travel factors could be responsible for part of the home advantage, provided that the journey involved crossing a number of time zones. However, as most of the matches does not require long travels, these factors were not thought to be a major cause of home advantage. It was evidence that the crowd factors appeared to be the most dominant cause of home advantage.
\newpar
\cite{Clarke} used least squares in order to fit a model to the individual match results in English football and produced a home field advantage effect on club level. Home advantages were calculated for all teams in the English Football League from 1981-82 to 1990-91, and some reasons for their differences were investigated. Their results showed no division effects but significant year effects. They found some evidence that teams with special facilities had significantly higher home advantage, and that London clubs in general had a lower home advantage. In addition, they found that the home field advantage had more leverage on winning than on goals margin. 
\newpar
\cite{Pollard} analysed over 400,000 sport matches since the start of the main professional sports, including English top division football. They found that the home advantage in English football decreased from the 1980s to the 2000s. Further, in contrary to \cite{Nevill} they claimed that travel and familiarity contribute to home advantage, but little in support of crowd effects. For association football, \cite{Pollard} home advantage is quantified as the number of points obtained by the home team expressed as a percentage of all points obtained in all games played. For the English Premier League, they provide results that annual values of home advantage of below 60 \% are not uncommon. 
\newpar

\section{Connections to Existing Literature}\label{Other_Relevant_Research}
Although this thesis focuses exclusively on optimizing a Fantasy Premier League team, parallels can be drawn to other scientific areas. The Fantasy Team Composition Problem shares features with many well-studied problems. However, these problems possess several significant differences with the FPLDP. 

\newpar

Knapsack problems share several similarities with the FPLDP. A capacity (purchase budget) has to be allocated to items characterized by a weight (purchase cost) and a reward (expected performance). \cite{Kirshner} modelled the problem of selecting free agent players for a NBA squad by using a multiple choice knapsack model. Here, the rewards are measured in terms of the ability of the players, weights as the salaries and the capacity as the NBA teams' salary cap. Furthermore, \cite{Gibson} used a stochastic knapsack problem in order to make team composition decisions regarding player drafts. 
Ahead of the American sports seasons, teams sign players according to a pick order. Therefore, players become stochastically unavailable over time, depending on the pick decisions of the teams. Thus, they model the draft decisions as a stochastic knapsack problem, where the future availability of the items is considered stochastic. There exist significant differences between the knapsack problem and the FPLDP. Firstly, the FPLDP includes not only adding items to a knapsack, it also considers removing items from the knapsack. This is done in terms of buying and selling players from your fantasy squad. In addition, removal decisions does not only free space to your knapsack, as it can also yield a reward in terms of increased or lowered budget. This is a result of stochastically changes of the player prices over time. 
\newpar
Similarities can be found between the FPLDP to that of staffing problems, where one seek to compose a set of personnel that satisfy the supply and demand of personnel from different categories \citep{Komarudin,Bruecker}. In general, FPLDP makes decisions in terms of buying and selling players, while staffing problems may consider hiring and dismissal of personnel. Although the two problems resembles each other in terms of composing a team, there exist significant differences. For instance, the FPLDP consider players as assets that can increase the value (budget) of the team, and not only as items used to satisfy a demand. \cite{Davis} aim to determine the optimal composition of the pre-hospital medical response team and evaluate the importance of including a doctor to the team. While the there is no distinction between workers in the same category for the staffing problems (e.g. individual nurses are assumed homogeneous), the players of in the Premier League are considered heterogeneous. For instance, although defenders are evaluated according to the point system of defenders, several factors have an impact of their expected performance. For example, the team of the player, his opponent team and his ability of scoring a goal, all influence the expected performance of a defender. 

\newpar

In capacity renewal problems \citep{Chand,Rajagopalan}, machinery are replaced due to damages over time, improved equipment and changes of production \citep{Hopp,Adkins}. Similarly, players are replaced in the FPLDP, but due to change in expected performance. This can be due to injuries, suspensions or as a result of poor recent performance. Hovewer, a significant difference between the problems is that while players are considered assets for maximizing a utility function (expected points) in the FPLDP, machinery is considered a necessary item in order to satisfy a demand in an efficient way. Furthermore, in the machinery problems, individual machines are often indistinguishable from one another. Hence, different machines can satisfy the same requirements. For the FPLDP however, players are considered unique as the expected performance is dependent on factors such as teammates, opponents and recent performance. factors such as teammates, opponents and recent performance. 

\newpar

For the long-term vehicle fleet composition \citep{Jabali}, different types of ships are needed in order to perform distribution activities, in which fleet size and mix of fleets are important decision variables. As for the FPLDP, a mix of players are necessary in order to fulfill the formation criteria. While different ships are selected in order to find an efficient solution, minimizing the costs, players are selected in the FPLDP according to a purchase budget. Various ships may contain the same characteristics and can hence be used for the same purposes. For the FPLDP however, although two players can be denoted by the same purchase price, their performance in terms of expected points can be significantly different. Furthermore, there is by far more uncertainty related to the performance of a football player, than to that of a ship's ability to satisfy a demand. 







\begin{comment}
\cite{Bell} studied team composition variables in order to increase team performance. The results of the meta-analysis performed can be used to effectively compose teams in organizations. \cite{Davis} aim to determine the optimal composition of the pre-hospital medical response team and evaluate the importance of including a doctor to the team. \cite{Jabali} present a continuous approximation model to determine the long-term vehicle fleet composition needed to perform distribution activities, in which fleet size and mix of fleets are important decision variables.  These problems resembles the Fantasy Team problem as similar decisions arise when one are to decide which formation one should use in addition to the mix of players. 
\end{comment}




\newpar

Finally, there is a large correspondence between FPLDP and portfolio optimization problems (see, e.g., \cite{Markowitz} and \cite{Speranza}). The FPLDP can be conceived as the problem of investing in a number of assets (football players) with stochastic returns in order to maximize some utility function (the total number of points). Furthermore, the budget limit in FPLDP can be seen as a budget constraint in portfolio optimization. However, in the FPLDP the decisions are binary (a player is either bought or not) while in portfolio optimization fractions of wealth can be allocated to different assets. Moreover, the fixed number of assets (players) in FPLDP resemble cardinality-constrained portfolio optimization problems \citep{Chang}. Additionally, while the formation criteria in the FPLDP gives a fixed number of players in each role, cardinality-constrained portfolio optimization problems may impose a minimum or maximum proportion of wealth to be allocated on certain class of assets. Finally, variance is a well known measure of risk in portfolio optimization problems and considering the resemblance to FPLDP, an implementation of a measure of risk should be considered. While variance is considered in terms of stock returns for portfolio optimization, it can somehow capture the uncertainty related to future player performances in the FPLDP. \cite{Dailyfantasysports} optimizes the expected payout of a tiered Daily NFL Fantasy Sports by using a stochastic integer program. Here variance is used as a measure of risk, and the standard deviation is approximated with a piecewise linear function. 

\section{Our Contribution}

\begin{itemize}
    \item første matematiske modellen/optimerings modellen for FPL 
    \item første frameworket for å løse FPL 
    \item se analogien mellom portfølje og FPL og inkludert portfølje varians constraints.
    \item første til å løse problemet med perfekt informasjon 
\end{itemize}




\iffalse %brukes for å kommentere ut alt nedenfor


\section{New Literature Study} \label{new} 

\subsection{An analytical approach for fantasy football draft
and lineup management}
The scope of the paper is to develop a methodology for Fantasy Football NFL that \textit{"predicts team and player performance based on the rich historical data, and builds a mixed-integer optimization model using such predictions for the draft selection as well as weekly line-up management that incorporates the entire objective of winning a fantasy football season"}. A difference worth noting is that in the NFL draft the managers have very limited time, so the speed of the model is more relevant than for us. Furthermore, most of the is effort devoted to handle the draft, and is not directly relevant. However, player performance for each week is predicted, and could be of relevance. It turns out the predictions are based on expert opinion of score over an entire season. Most of the paper is focused on distributing these points over each gameweek. Therefore, it is not directly relevant. 
\newpar

\subsection{The Football Team Composition Problem: A Stochastic Programming approach}
\textit{In this paper we consider a football club’s strategic problem of investing an available budget to purchase football players. The scope is that of maximizing the expected value of the team - represented by the sum of the transfer market appreciation of the players in the team}. The problem is referred to as the Football Team Composition Problem (FTCP). In some sense a motivation for our thesis.

\subsection{Valuing Individual Player Involvements in
Norwegian Association Football}

3 research questions are formulated:
\begin{enumerate}
    \item Is it possible to create a statistically significant xG model that assesses the quality of all shots in Tippeligaen in order to evaluate the efficiency of primarily offensive players?
    \item Is it possible to create a Markov game model for football that is able to evaluate all player involvements in a match and rate players over the course of a season?
    \item Is it possible to reveal undiscovered talent or identify under- and overvalued players based on the evaluation of individual player involvements?
\end{enumerate} 

\textbf{Litterature}\newline
In McHale et al. (2012) this index is described, and it consists of six subindices from which individual players get scores: match contributions, winning performance, match appearances, goals scored, assists and clean sheets.\newpar

All in all, what appears to be most relevant is the literature review, and perhaps how the thesis is built up. The methodology is not directly relevant.

\subsection{Predicting the Future Performance of Soccer Players}
Propose a multitask, regression-based approach for predicting future performances of soccer players. Appears to be to difficult.

\subsection{Mathematical programming as a tool for virtual soccer coaches: a case study of a fantasy sport game, Bonomo et al(2014)}

This paper is highly relevant to our project, as their objective is not so different from ours. Their objective is to develop is to develop two mathematical programming models that act as virtual coaches that choose a virtual team lineup for each round of the Argentinian soccer league. The two models are called \textit{a priori} and \textit{a posteriori}. \textit{A priori} creates a competitive team for each game taken into consideration which transfers to complete, while the \textit{a posteriori} determines what would have been its optimal lineup once the season is over and we have the perfect information. 

\newpar

The points are allocated in the same way, both objective statistics(goals scored, shutouts, yellow and red cards shown) and subjective statistics(individual player performance in each match). The subjective statistics can resemble the Bonus Point System in Fantasy Premier League. To their knowledge there are no optimization or mathematical programming algorithm applications of the sort proposed in the present work that provide real support for real or virtual coaches. Some of the game rules differs from FPL: 
\begin{enumerate}
    \item A player that is on the field for less than 20 minutes of a match is deemed not to have played that match and hence substituted by a substitution player in the same position. In FPL, if a player plays a minute in a game, everything from 1 to end of match, is regarded as one of the 11 playing players. 
    \item The amount of transfers in each round is maximum four, while in FPL the amount of free transfers given each week is 1 and maximum accumulated free transfers is 2. It also has the possibility of doing point-punishing transfers. 
    \item Seems like you have to choose the team formation in advance, while in FPL the team formation is adjusted afterwards according to initial formation chosen before the gameweek is played and after taken into account if any player was substituted. 
\end{enumerate}

They conclude that using player's average in recent rounds does not give a predictor of the points, and hence some factors were weighted and included. A player's point prediction was calculated by taking the point average in the rounds already played and weighted by three factors: 
\begin{enumerate}
    \item home/away status of next round match(1.05 for home games and 0.95 for away games) 
    \item league table position of the next round competitor(1 to 1.05 if in the bottom five of the table, 0.95 to 1 if in the top five)
    \item current performance or situation of the player or the club (up to 5 \% more if in a scoring or winning streak, respectively; up to 5 \% less if on a scoreless or winless streak, or tired after, for example, a recent league or international match)
    \item "starting lineup" factor. This factor is 1 for those players assumed to be starters in the next round(based on coaches' announcements, press reports, or information posted on the game website) and 0 for all other players. 
\end{enumerate}

The second-last factor seems like a rather subjective assessment, and it should be quantified what it means to be "tired". For example that a player is labelled "tired" if they play more than \textit{X} minutes in \textit{Y} gameweeks. After testing various factors, they use a weight factor of 0.1 to substitutes in the objective function. The same could be applied to our optimization model and test for various cases. By such, one could determine beforehand the level of importance of the substitutes. Setting high weight factor means the model would choose more good performing substitutes, and setting the weight factor low implies choosing more low performing substitutes.   

\newpar

The \textit{a priori} model generates the initial team by maximizing the total expected points in the first round of the game. This is what is called a "myopic initial team". The model also runs for a "non-myopic initial team" where the initial team is based on the first three gameweeks. The "non-myopic initial team" gives better results, which suggests that there are value in future data. The possible drawback is that the model do not use real scores, but predictions which may be increasingly inaccurate the later the round is considered. 

\newpar

By fixing a formation which is held throughout the whole season, it is possible to find which formation gives highest points. The best formations are a 1-4-3-3 and 1-3-4-3. Their results places them among the tip 0.2 \% in of the whole tournament. They suggest an interesting use of such tool, by saying that this model could be used as a complementary support tool. It could in each round propose the \textit{k} transfer for the team in each round and the expert could then choose among them. This could easily be done, by running the model \textit{k} times and each time restrict the previous optimal solution.  

\begin{comment}
De skriver hvilke formasjoner som er mulig å bruke. Det burde vi også gjøre. 
\end{comment}

\subsection{Optimization modelling for analyzing fantasy sport games, Beliën et al(2013)(Står i dokumentet MathSport2013Proceedings)}

This paper presents an mixed integer programming model for finding an optimal set of decisions in fantasy sport games under the assumption that all data are known beforehand. Thus, the model allows to identify ex post an optimal team selection and player transfers. They argue that an ex-post analysis gives both \textit{information value} and \textit{commercial value} of the results. The \textit{information value} is that it would answer questions to readers and fans such as "what is the optimal team of this week?", "what is the optimal team so far?" and not only the top X best performing teams of the specific gameweek. The \textit{commercial value} is argued to be that by showing the big difference in points by perfect team and the best performer, that it would attract more participants. 

\newpar

Further, it discusses the common characteristics in fantasy sports games where the common game rules characteristics are most relevant. A table of the main characteristics is presented which includes factors such as budget, player distribution, fantasy team composition and calculation of points. A MIP model is presented for finding optimal ex-post decisions in fantasy sport games. The objective function maximizes the points collected collected over all games. In case several solutions exist with maximal points, the solution with the highest remaining budget in the final period is selected.  With modification, the model can be used for FTCP. However, there are some main differences: 

\begin{enumerate}
    \item The concept of substitution priority is not considered, but only one substitution per player position is modelled. In Fantasy Premier League it is possible to have more than one substitution per player position, hence the need of modelling substitution priority. 
    \item The inclusion of the game chips is something extraordinary for FTCP. 
\end{enumerate}


\begin{comment}
\textbf{INSPO}

INTRODUCTION 
A fantasy sport game allows ordinary people to act like a team manager building a team of real individual
professional sport athletes. The real-world performances of these athletes (or their teams) are translated into
points for their team managers. The managers’ aim is to collect as many points as possible thereby defeating
the fantasy teams of opponents. In the remainder of this paper we will use the word 'participant' to describe a
virtual team manager while the word 'player' solely refers to a real-world athlete.
Fantasy sports found their origin in the United States in the 1980s. American journalists Glen Waggoner
& Daniel Okrent developed a game in which a handful of participants would draft from a pool of active
baseball players (Davis & Duncan, 2006). From the 1990s on, the Internet made game results more
accessible, and virtual leagues easier to manage. But fantasy sports really took off around the turn of the
millennium, when the internet transformed fantasy sports into a mainstream phenomenon. Fantasy sports are
now being played by tens of millions of people worldwide. Fantasy American football has become the most
popular fantasy sport league, with a market share of 80% in the United States and Canada
(http://www.fantasysportsadnetwork.com). In Europe, soccer is the most popular fantasy sport subject. The
enormous success of fantasy sports has a lot to do with the fact that it allows online participants to assume an
active role of a team owner or a team manager in a sport they are heavily interested in, thereby intensifying
the way live sport is consumed. As a result, many sports enthusiasts are now obtaining their sports
entertainment through fantasy sports rather than solely by watching games on television (Nesbit & King,
2011).

LITERATURE STUDY 
The growth of fantasy sports has made it an important part of the sports industry. The booming popularity
also stimulated research on fantasy sports and from 2005 on literature on fantasy sports really started to
grow. Two directions of research are currently dominant. The first line of research is economically oriented.
It sees fantasy sports as a new form of sport consumption and studies how this affects the behavior of sports
fans. Randle & Nyland (2008), Drayer et al. (2010) and Karg & McDonald (2011) all take a rather global
view and look at the impact of fantasy sport participation on the various media sources fantasy sport
participants use. The specific impact of fantasy sport on television ratings is analyzed by Nesbit & King
(2010), Dwyer (2011a) and Fortunato (2011), while the live attendance impact is taken up by Nesbit & King
(2011). Most of these studies basically conclude that instead of competing with traditional ways of sport
consumption, fantasy sport appears to be a complementary and value-adding activity (Dwyer, 2011a). Fan
loyalty and how fantasy sports can be used in customer relationship management are studied by Dwyer
(2011b) and Smith, Synowka & Smith (2010).
\end{comment}


\subsection{Optimizing Tiered Daily Fantasy Sports - Mathematically Modelling DraftKings NFL MIllionaire Maker Tournament, Sarah Newell et al(2017)}

The problem described in this paper resembles FPL, with the main difference that a Daily Fantasy Sports lasts for a single day rather than a full season. Hence, transfers are never considered in their model. It gives the first model to optimize the expected payout of a tiered DFS contest through a stochastic integer program. They also discuss some interesting metrics on how much money participants spend on decision tools. An estimated 30 \% of fantasy sports participants use additional websites to research athletes and other factors. Together, these participants spend \$ $656$ million annually to purchase additional information and decision-making tools. 

\newpar

In the model each player is assumed to be independent random variable from a normal distribution, and consequently the team is normally distributed with mean and variance that is the sum of each athlete's mean and variance. The team's standard deviation is estimated through a piecewise linear function of the team's variance. 
\begin{comment}
INSPO: 
Approximately 56.8 million people worldwide played fantasy sports in 2015 [1]. The fantasy sports industry is
expected to grow annually by 41% and generate $14.4 billion by the year 2020 [2]. The large number of fantasy sport
participants has impacted the sports industry positively, as research shows that fantasy sport participation increases
game attendance and sports media viewership [3].
\end{comment}


\subsection{Further adjustments}

\begin{enumerate}
    \item Optimization models: 
    \begin{enumerate}
        \item \textbf{MathSport2013Proceedings.pdf(side 21-29)}. Artikkel som handler generelt om fantasy sport games- 
        \item \textbf{Optimizing Tiered Daily fantasy sports –Mathematically Modeling DraftKings® NFL Millionaire Maker Tournament}. Artikkel som handler om NFL. Her snakker de også om varians.
        \item \textbf{Mathematical programming as a tool for virtual soccer coaches: a case study of a fantasy sport game}. Artikkel om argentinsk liga. 
        \item \textbf{Matthews et al(2012)}. Artikkel som bruker Bayesian Q-learning på FPL.
        \item Hvor kommer Fry og Gibson inn, og trenger man dem? 
    \end{enumerate}

    \item Scoreline Prediction
    \begin{enumerate}
        \item korte ned det som allerede står der og deretter fortelle at man prøvde en slik metode i prosjektoppgaven uten noe hell. 
    \end{enumerate}
    
    \item xG model 
    \begin{enumerate}
        \item dette kapittelet er kun med hvis xG brukes i average metoden. Må justeres da.
    \end{enumerate}
    
    \item Fantasy Blogs
    \begin{enumerate}
        \item Slette
    \end{enumerate}
    
    \item Forecasting 
        \begin{enumerate}
            \item finne regresjon på fantasy sports 
            \item Artikkel \textbf{Predicting the Future performance of
            soccer players}, Artikkel \textbf{Vabo} og artikkel \textbf{Predicting NBA Results based on 20 past season. Player performance statistics using R.}
        \end{enumerate}
\end{enumerate}
\fi  %knyttet til \iffalse