%===================================== CHAP 1 =================================

%Only for right page counting---------------
\pagestyle{fancy}
\fancyhf{}
\renewcommand{\chaptermark}[1]{\markboth{\chaptername\ \thechapter.\ #1}{}}
\renewcommand{\sectionmark}[1]{\markright{\thesection\ #1}}
\renewcommand{\headrulewidth}{0.1ex}
\renewcommand{\footrulewidth}{0.1ex}
\fancyfoot[LE,RO]{\thepage}
\fancyhead[LE]{\leftmark}
\fancyhead[RO]{\rightmark}
\fancypagestyle{plain}{\fancyhf{}\fancyfoot[LE,RO]{\thepage}\renewcommand{\headrulewidth}{0ex}}

\pagenumbering{arabic} 				
\setcounter{page}{1}
%--------------------------------------------

\chapter{Introduction}\label{introduction}
Analysis of football matches has been an important aspect of the game for several years. Typical applications of such analysis are to get information about the opponent teams, players in the transfer market as well as information about composition of a team's existing players. Analysis are performed both before, during and after matches are played. Analysis have previously been qualitative in nature. However, nowadays quantitative analysis has gained momentum, and companies like Opta and Sportradar make profit on delivering sophisticated data sets to professional football clubs and bookmakers. In combination with the steady increase
in computational power, these data sets create new possibilities for statistical analysis in sports,
commonly referred to as sports analytics.
\newpar
For the past decade, a phenomenon called Fantasy Sports has experienced an enormous growth, both by terms of active players and by media attention. Fantasy Sports is a type of online game where participants assemble imaginary teams consisting of real players within a professional sport. Fantasy teams compete in both private- and public leagues, where the teams gain points based on the statistical performance of real players in actual games. According to Fantasy Sports Trade Association (2017), it is estimated that 59.3 million players in the U.S and Canada are competing in some kind of Fantasy Sports. Fantasy Premier League (FPL) the Fantasy Sports of England's top division football league, Premier League. It is the largest Fantasy Sports in Europe, with more than 5.5 million active players. 
\newpar
Fantasy Sports has its origin from the latter part of 1950s when a limited partner of the Oakland Raiders, Wilfred Winkenbach, invented a Fantasy Sports for golf. Each player selected a team of professional golfers, and the player who suggested the team with the lowest total score would win the tournament. In 1962, Winkenbach suggested the first fantasy league for American Football, initiating the birth of what has become a billion-dollar business. Nowadays, Fantasy Sports are available for several sports, including American football, basketball, ice hockey, baseball, football and cricket.  
\newpar
Typically, a Fantasy Sports team consists of selecting a squad consisting of a given amount of players in each position on the field. Each real-life player is given a buying price based on their skill level and ability to perform well in the Fantasy Sports. The performance of each player in the actual games is measured in terms of points, where an action like scoring a goal will provide the player with a specified amount of points. In order to avoid players from selecting only top rated real-life players, each manager is typically given a restricted budget.
\newpar
For the past 10 years, a new phenomenon called Daily Fantasy Sports (DFS) has arisen. DFS is an alternate of the original Fantasy Sports, allowing players to compete for short time-periods, typically for one day or a weekend. DFS is structured in a way where managers pay a fee for participating in a competition and the winners receive a predetermined amount of the entry fees. In the U.S, FanDuel and DraftKings are the main providers of DFS, with both companies having a worth of more than \$1 billion.

%The two companies nearly collected a combined entry fee of \$3 billion in 2015. For instance, DraftKings arranged a 1-day contest for the Major League Baseball in 2015, with combined winning prizes of \$500 000 where the first-place finisher received \$125 000.%
\newpar
Although Daily Fantasy Sports are primarily available for American leagues, FanDuel and DraftKings were granted permission to launch a DFS for the English Premier League in 2015 \citep{Purdum}. With the great interest of the English Premier League, and the amount of active FPL managers, a question has arisen of why DFS is not that big in Europe. The answer lies in the rules of gambling in the US and in Europe. While sports gambling is illegal in most of the states in the US, sports gambling has been a tradition in Europe for several years. Hence, European gamblers might not consider Daily Fantasy Sports as a substitute for regular sports gambling.  
\newpar
For the past years, there has been lack of European bookmakers providing gambling for fantasy leagues. In 2017 however, Unibet, a Scandinavian bookmaker, launched its first Daily Fantasy Sports, Fantasy Sports Beta, allowing their customers to compete for money within the English Premier League. In addition, a Norwegian bookmaker GUTS, included bets on Fantasy Premier League in 2016. For instance, one put money on whether one player will perform better than another for the upcoming week. In addition one can bet on which of the top rated players to perform best for the upcoming gameweek. As a result of Fantasy Premier League becoming more popular over years and European bookmakers providing odds for the game, it may seem like Fantasy Sports can experience a growth in Europe as well. 

%According to Sportradar, a Norwegian company providing data for international bookmakers, there seem to be an increasing interest for the subject. Although money-making on Fantasy Premier League is rather insignificant at the time, Sportradar assume that Fantasy Sports can be an important aspect of bookmakers' offer in the future. 
\newpar
Fantasy Premier League managers usually select their teams based on their subjective opinion, common knowledge, tips suggested by football experts, blogs and discussion forums. In 2006 \cite{Smith} explored a discussion forum for NBA  Fantasy containing more than 82,000 messages in order to examine Fantasy managers' strategies. These forums primarily consisted of managers sharing their line-up and discussing which players to select. By analyzing the content of the forums, the authors found that managers mainly determine their fantasy sports rosters by terms of reputations, team loyalty or personal predictions of future performance. In addition, some statistics such as average points and assists were often presented in the threads. However, these statistics were seldom discussed and never in terms of statistical analysis. 
\newpar
The main objective of this project is to create an optimization model which uses statistical data for the English Fantasy Premier League, in order to decide which players the team should be composed of. Hence, the problem is regarded as the \textit{Fantasy Team Composition Problem} (FTCP). As far as we know, there are not similar models produced, and therefore we aim to create an optimization model which supports football experts' intuition and knowledge.
\newpar
As there exist several optimization models for ice hockey \citep{drafting_hockey_pools} and American football\citep{Fry}, the main objective of this project is to produce a competitive model for Fantasy Premier League as well. Compared to the models suggested for the mentioned Fantasy Sports, there is a difference when modelling a Fantasy Premier League team. While the classical American Fantasy leagues consist of selecting a team for an entire season and not allowing for transfers, the Fantasy Premier League allows for making transfers of players during the season. Thus, FPL managers has to consider making new decisions each gameweek in a season. Therefore, this project propose a model for making optimal decisions in a multi-week manner, not only at the start of a season or for just one gameweek as in the Daily Fantasy Sports. To our knowledge, this is the first model suggested for making fantasy decisions in a multi-week manner.
\newpar
Many readers may be interested in detailed analysis, providing answers to questions like "what is the optimal team so far?" and "what are the optimal transfers for each particular gameweek?". These answers can easily be answered by an ex-post optimization model. Therefore we find it interesting to run an optimization model under the assumption that all data are known beforehand. In addition, the model can then be used to determine which were the optimal transfers in a particular gameweek for a given participant's starting team. Further, it would be interesting to compare the optimal solution to the score of the manager who finished on top of the overall rankings. It is quite reasonable that the optimal solution strategy largely outperforms that of the best manager. Thus, when realizing how big the gap between an optimal solution and the winning team is, people get the impression that winning FPL is not that difficult after all and hence overestimate their chances of winning the game. Consequently, computing an optimization model based on perfect information can attract more participants to the game.
\newpar
In addition, the approach suggested in this thesis can be generalized to apply for other situations facing decision-making similar to those in FPL. For instance, the team composition model may be of inspiration to actual football teams when considering which players to sign during a transfer window. Even though some adjustments should be made, the characteristics of the problems are related in a great manner. As for FPL, a football team has got a restricted budget when purchasing new players. While FPL managers base their transfers on projected points gained by a player, football managers purchase players they expect to perform well in the future. By further comparison, FPL consists of a set of players to select, while the actual clubs can buy players from a similar set of players, namely the set of all players available in a transfer window.
\newpar
The project is structured as follows. Chapter 2 provides the problem description. Here, the rules and point system of Fantasy Premier League are presented. Chapter 3 provides a structural literature review of Fantasy Sports optimization models, scoreline prediction and measuring of player performance. Chapter 4 describes the mathematical formulation of the problem. It includes the problem structure, assumptions as well as a mathematical model for the Fantasy Team Composition Problem. Chapter 5 describes the solution approach used in this project. Chapter 6 describes the experimental setup of the approach as well as it introduces the data sets used in order to suggest a solution. Chapter 7 provides the results obtained by the optimization model and compare them to the weekly rankings of FPL. Chapter 8 gives a conclusion of the work that has been done. Finally, Chapter 9 addresses the potential for further research.

% \subsection{Further adjustments}

% \begin{enumerate}
    % \item få inn motivasjon om at det er flere bettingselskaper som har begynt å legge inn odds på fantasy pl 
    % \item få inn mer konkret fra sportsradar om hvorfor fpl er interessant. ergo noe på at kunder etterspør mer odds der i fra etc.. 
    % \item få inn et avsnitt om hvor stor feltet sport analytics har blitt, ergo hvor stor budjsett klubber har på analyse og hvordan dette har utviklet seg de siste ukene. 
    % \item knytte hvorfor dette problemet er interessant utenom fantasy. 
% \end{enumerate}

% rekkefølge: 
% \begin{enumerate}
    % \item sport analytics 
    % \item fantasy. Ta med artikkel om hvordan valg i fantasy communitites blir tatt. Se på drive.  
    % \item sportradar 
    % \item hvorfor dette problemet er interessant for ikke-fantasy folk
    % \item strucuture på rapporten
% \end{enumerate}