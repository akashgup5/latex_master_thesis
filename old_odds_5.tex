\subsection{Prediction using odds}
The third approach for player point prediction is by use of odds created by bookmakers. As stated in the literature review in Chapter 3, \cite{Hvattum} found that when used for forecasts, Elo ratings appeared to be considerably less accurate than market odds. Therefore, we find it interesting to adapt an approach based on bookmakers odds.

It is reasonable to assume that bookmakers provide a realistic odds distribution for the different scorelines, in order to avoid profitable gambling. By acquiring the result outcomes for each Premier League match, a scoreline prediction can be obtained. By summing all the result odds and dividing each individual result odds by that sum, one can obtain the probability of an exact result to occur. A great drawback with this approach, is the fact that bookmakers only provide odds for the upcoming gameweek. Hence, the odds can only be used in order to predict the fantasy points one gameweek ahead. 

\newpar

\begin{comment}

\end{comment}

\textit{Få med avsnitt om hvordan hva som ligger bak scoreline prediction og expected goals, referer til Maher etc.}

\newpar


\newpar

When the scoreline predictions are acquired by use of result odds, one has to link these predictions to player performances. In order to determine the probability for a defender or midfielder to keep a clean sheet, one simply sum the probabilities of each match outcome where its team does not concede a goal. Further, bookmakers provide odds' for each individual player to score a goal or have an assist in a particular match. By summing the goalscorers odds, one can obtain the cumulative probability distribution of the goalscorers. Once the cumulative distribution is obtained, one simply multiplies the goalscoring probabilities with the result probabilities in order to assign goalscoring points for each player. As for the assist, a similar approach may be used. However, some of the goals scored are unassisted which must be taking into account when predicting the assist points. This can be done by multiplying a players assist points with a factor representing the amount of goals being assisted. Finally, as for the yellow cards, one simply use the odds in order to calculate the probability of a player receiving a yellow card. For instance, if a player has a probability of 0.3 for receiving a yellow card in a particular match, his expected points is penalized with -0.3 as a yellow card deducts 1 point. 
\newpar
The largest bookmaking companies offer odds for all the items mentioned in the paragraph above. As goals scored are the greatest point factor in Fantasy Premier League, this area is of great interest. As mentioned, it is necessary to obtain the probabilities for each result outcome in a given match. The following example provides Unibet's result odds for a match-up between Leicester and Burnley on April 14th 2018. 

\begin{table}[H]
\centering
\caption{Result odds for Leicester-Burnley 14.04.18}
\label{Leicester-Burnley}
\begin{tabular}{|ll|ll|ll|}
\multicolumn{6}{c}{Odds}                   \\
\hline
1-0 & 7.00   & 0-0 & 7.50   & 0-1 & 8.00   \\
2-0 & 11.00  & 1-1 & 6.00   & 0-2 & 13.00  \\
2-1 & 9.50   & 2-2 & 15.00  & 1-2 & 10.50  \\
3-0 & 23.00  & 3-3 & 67.00  & 0-3 & 29.00  \\
3-1 & 21.00  & 4-4 & 301.00 & 1-3 & 23.00  \\
3-2 & 31.00  &     &        & 2-3 & 35.00  \\
4-0 & 61.00  &     &        & 0-4 & 81.00  \\
4-1 & 56.00  &     &        & 1-4 & 67.00  \\
4-2 & 81.00  &     &        & 2-4 & 91.00  \\
4-3 & 181.00 &     &        & 3-4 & 201.00 \\
5-0 & 201.00 &     &        & 0-5 & 276.00 \\
5-1 & 181.00 &     &        & 1-5 & 226.00 \\
5-2 & 276.00 &     &        &     &        \\
\hline
\end{tabular}
\end{table}

The odds provided in table \ref{Leicester-Burnley} can be translated into match result probabilities. This is done by summing all the inverse odds and afterwards dividing each inverse odds by the sum. The results are shown in table \ref{Prob.Lei-Bur}.


\begin{table}[H]
\centering
\caption{Probabilities for Leicester-Burnley 14.04.18}
\label{Prob.Lei-Bur}
\begin{tabular}{|ll|ll|ll|}
\multicolumn{6}{c}{Odds}                         \\
\hline
1-0 & 0.104388 & 0-0 & 0.097429 & 0-1 & 0.09134  \\
2-0 & 0.066429 & 1-1 & 0.121786 & 0-2 & 0.056209 \\
2-1 & 0.076918 & 2-2 & 0.048714 & 1-2 & 0.069592 \\
3-0 & 0.03177  & 3-3 & 0.010906 & 0-3 & 0.025197 \\
3-1 & 0.034796 & 4-4 & 0.002428 & 1-3 & 0.03177  \\
3-2 & 0.023571 &     &          & 2-3 & 0.020878 \\
4-0 & 0.011979 &     &          & 0-4 & 0.009021 \\
4-1 & 0.013049 &     &          & 1-4 & 0.010906 \\
4-2 & 0.009021 &     &          & 2-4 & 0.00803  \\
4-3 & 0.004037 &     &          & 3-4 & 0.003635 \\
5-0 & 0.003635 &     &          & 0-5 & 0.002648 \\
5-1 & 0.004037 &     &          & 1-5 & 0.003233 \\
5-2 & 0.002648 &     &          &     &         \\
\hline
\end{tabular}
\end{table}

Once these probabilities are obtained, it is possible to calculate the expected goals scored by Leicester. This is done by multiplying the probability of a result occuring by the amount of goals scored by Leicester in that particular result. Further, one have to sum all the probabilities in order to find Leicester's expected goals scored. In the example from table \ref{Prob.Lei-Bur} Leicester's expected goals is found to be 1.234. 
\newpar
Further, it is necessary to assign expected goalscoring- and assist points to each player. For instance, if Jamie Vardy has a cumulative probability of 0.26 of scoring a goal for Leicester, his expected goalscoring odds is found by multiplying 0.26 with the expected goals scored by Leicester. 
\begin{equation*}
    \textrm{Jamie Vardy expected goals scored} = 1.234 \times 0.26 = 0.321
\end{equation*}

Finally, his expected points gained from scoring goals is calculated according to the Fantasy Premier League point system: 

\begin{equation*}
    \textrm{Goal points Jamie Vardy} = 0.321 \times \textrm{4 points} = \textrm{1.284 points}
\end{equation*}

A similar approach can be used in order to calculate a player's expected points from assists. However, one have to consider the fact that not all goals are assisted: 

\begin{equation*}
    \textrm{Expected assist points} = \textrm{Expected goals} \times \textrm{Player's assist probability} \times \textrm{Assist probability}
\end{equation*}

As for the clean sheets, the probability of a team keeping a clean sheet is found by summing the probabilities of all the result outcomes were the team does not concede a goal. According to table \ref{Prob.Lei-Bur} Leicester has got a probability of 0.316 for keeping a clean sheet. Hence, the starting defenders and the  goalkeeper are expected to gain

\begin{equation*}
    0.316 \times \textrm{4 points} = \textrm{1.264 points}
\end{equation*}
from keeping a clean sheet. 

\newpar
In addition, one can use odds for yellow cards in order to calculate a player's expected deducted points for receiving a yellow card. On the bookings sites each player is listed with an odds for a probability of being booked. The probability of receiving a yellow card is simply the inverse of the odds. For instance, if a player is listed with an odds of 3.00 for receiving a yellow card, it is a probability of

\begin{equation*}
    \frac{1}{3.00} = 0.333 
\end{equation*}

that the player receives a yellow card. Hence, his expected points is deducted with:

\begin{equation*}
    0.333 \times \textrm{1 point} = \textrm{0.333 points.} 
\end{equation*}
